\section*{Introduction}

PBErr is a C library providing structures and functions to manage exception at runtime.\\ 

It uses no external library.\\

\section{Interface}

\begin{scriptsize}
\begin{ttfamily}
\verbatiminput{/home/bayashi/GitHub/PBErr/pberr.h}
\end{ttfamily}
\end{scriptsize}

\section{Code}

\subsection{pberr.c}

\begin{scriptsize}
\begin{ttfamily}
\verbatiminput{/home/bayashi/GitHub/PBErr/pberr.c}
\end{ttfamily}
\end{scriptsize}

\section{Makefile}

\begin{scriptsize}
\begin{ttfamily}
\verbatiminput{/home/bayashi/GitHub/PBErr/Makefile}
\end{ttfamily}
\end{scriptsize}

\section{Unit tests}

\begin{scriptsize}
\begin{ttfamily}
\verbatiminput{/home/bayashi/GitHub/PBErr/main.c}
\end{ttfamily}
\end{scriptsize}

\section{Unit tests output}

\begin{scriptsize}
\begin{ttfamily}
\verbatiminput{/home/bayashi/GitHub/PBErr/unitTestRef.txt}
\end{ttfamily}
\end{scriptsize}

testio.txt:\\
\begin{scriptsize}
\begin{ttfamily}
\verbatiminput{/home/bayashi/GitHub/PBErr/testio.txt}
\end{ttfamily}
\end{scriptsize}

